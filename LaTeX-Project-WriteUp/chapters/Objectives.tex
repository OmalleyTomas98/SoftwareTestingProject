\subsection{Objectives}
Here some of the primary  objectives of the test Plan
\begin{enumerate}
 
  
	\item Defining Tasks and Responsibilites 
   \begin{itemize}
     \item One of the main areas of the test plan is to precisely define the tasks and the responsibilities of each member involved in the company and mark the Software testing life.
   \end{itemize}


\item Communication 
   \begin{itemize}
     \item Communication is vital when working in a team and can be a barrier to successful testing. The test plan allows the team to know the issues e.g Unit Testing bug and can allow other members to collaborate when appropriate.
   \end{itemize}


\item Documentation 
   \begin{itemize}
     \item Overall this test plan is being a documentation mapping the continuous integration and or progress of the Software product we are testing. This document will allow us to record the overall quality of the Companie products and will allow us to excel in the manner we overcome/analyze problems in future applications. 
   \end{itemize}


   
\end{enumerate}


\subsection{Tasks}
Here are some of the tasks that I will go into detail later


\begin{enumerate}
   \item Report Document
   \begin{itemize}
     \item  Each member of the team is required to report the obstacles they encountered during their interaction with the software lifecycle process.
   \end{itemize}
  
  \item Create Validation
   \begin{itemize}
     \item Validation refers to the action of checking or proving the validity or accuracy of something. We aim to validate each component in the Product for example character attributes such as health using testing strategies such as unit testing etc.
   \end{itemize}

     \item Test Products Components
   \begin{itemize}
     \item Each component of the product will be marked to a tester and they will be responsible for documenting their discoveries and including files such as .csv mapping the time and the component where the program passed or failed.
   \end{itemize}


   \item  Bug  Report
   \begin{itemize}
     \item A bug report must be created and maintained throughout the testing process. Bug report systems such as Bugzilla will be used to optimize our productivity and diagnose the severity of the faults in the program.
   \end{itemize}

 

  \item Testing methodologies
   \begin{itemize}
     \item The testing methodologies will be defined by the complexity and the severity of the Product attribute. For example if the game has an issue with users losing their save games this is more vital than in-game health issues. We must access what are the most important components e.g user security.
   \end{itemize}
\end{enumerate}
