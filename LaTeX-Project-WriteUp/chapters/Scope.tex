Game Development International Ltd has a rich catalog of features in their game which I cover briefly in
the Introduction.
\subsection{General}
Here are some of the technical features
\begin{enumerate}
  \item    \textbf{Menu Systems :}  Test the products menu such as the start  pause menu screens
  \item    \textbf{In game controls :} Test the basic used to interact with the game e.g W- To move forward
  \item     \textbf{Character Attributes :} Test the basic character attributes such as health and points.\\


  All of the features listed are also interfaces allow the user to interact with the 2D game and set the foundation for the game.Test cases will be designed and allocated to developers to througlhy test the intergation of each componet/module.
\end{enumerate}


\subsection{Tactics}


All tests will be carried out in Alpha Testing Environment.


Alpha testing is a type of acceptance testing; performed to identify all possible issues/bugs before releasing the product to everyday users or the public. Alpha testing is carried out in a lab environment and usually, the testers are internal employees of the organization.


The Scope of this approach is to find as many
bugs in the game and understand how these bugs congregate in other areas of the application.
To test all of these existing functions I will be incrementally testing using a procedural approach for all, An example. When testing the Save Game
I would note the steps taken e.g 

\begin{enumerate}
	\item  Launch game on browser/Unity Etc
	\item  Click PLay Button
	\item Press the pause button in game e.g ESC
	\item  Click the Save Gamwe button listed using mosue or Arrwo keys and press enter button on keyboard
	\item Exit Game by pressing exit button using arrow keys and enter keys 
	\item Click Resume button
	\item If game resumes from save file . Test case is a Pass if not , a fail.

\end{enumerate}

This is a very specific example but the incremental step , dumb user approach is the most successfully method as we can accurately test the links from small modules to larger modules.